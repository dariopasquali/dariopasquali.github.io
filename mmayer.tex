
\documentclass[10pt,a4paper,ragged2e]{altacv}
\geometry{left=2cm,right=10cm,marginparwidth=6.8cm,marginparsep=1.2cm,top=1.25cm,bottom=1.25cm}
\ifxetexorluatex
  \setmainfont{Carlito}
\else
  \usepackage[utf8]{inputenc}
  \usepackage[T1]{fontenc}
  \usepackage[default]{lato}
\fi
\usepackage[utf8]{inputenc}
\definecolor{VividPurple}{HTML}{000000}
\definecolor{SlateGrey}{HTML}{2E2E2E}
\definecolor{LightGrey}{HTML}{2E2E2E}
\colorlet{heading}{VividPurple}
\colorlet{accent}{VividPurple}
\colorlet{emphasis}{SlateGrey}
\colorlet{body}{LightGrey}
\renewcommand{\itemmarker}{{\small\textbullet}}
\renewcommand{\ratingmarker}{\faCircle}
\addbibresource{sample.bib}

\usepackage{fancyhdr}
\pagestyle{fancy}      % Set the page style to fancy
\fancyhf{}             % Clear the default header and footer
\fancyfoot[C]{\small \textit{Autonomously generated from my website content using a Continuous Integration GitHub Pipeline.}}  % Add a small footer at the center

\usepackage{ragged2e}
\begin{document}



\name{DARIO PASQUALI}
\tagline{Post Doc @ Istituto Italiano di Tecnologia}

\photo{3.3cm}{me3.jpg}
\personalinfo{
  \email{dario.psquali@iit.it}
  \email{dario.pasquali93@gmail.com}
  \phone{(+39) 324 7956801}
  \location{Bologna - Italy}
  \homepage{https://dariopasquali.github.io/}
  \linkedin{dario-pasquali}
}

%% Make the header extend all the way to the right, if you want.
\begin{fullwidth}
\makecvheader
\end{fullwidth}


\AtBeginEnvironment{itemize}{\small}
\cvsection[page1sidebar]{About Me}
Experienced Software Engineer specializing in creating end-to-end user interaction systems that integrate Deep Learning with low-level devices.

\cvsection{Work Experience}

\cvevent{Post Doc}{COgNiTive Architectures for Collaborative Technologies (CONTACT)}{Jul 2022 -- Now}{Istituto Italiano di Tecnologia }
    Actively working on National (\textbf{FAIR}) and European (\textbf{ARIEL}) research projects. Reference person in the laboratory for architectural and Software Engineering matters. Tutoring bachelor and master internship students, and mentoring PhD candidates.

    \divider    
    
    \cvevent{Visiting Research}{Social and Intelligent Robotics Research Lab (SIRRL)}{Jul 2021 -- Nov 2021}{University of Waterloo }
    Development of a textual adventure to challenge players against Social Engineering threats. Real-time control of the humanoid robot Furhat. Multi-modal acquisition and processing of physiological data from an Eyelink 1000, a Tobii Pro Glasses 2 and a Shimmer3 GSR+, used to predict humans' compliance. Evaluation of different intervention strategies for a robot to prevent users' compliance.

    \divider    
    
    \cvevent{Ph.D Candidate}{Robotics, Brains and Cognitive Sciences (RBCS) \& ICT}{Nov 2018 -- Jul 2022}{Istituto Italiano di Tecnologia }
    Machine-learning based real-time evaluation of pupillometry, heart rate and electrodermal activity to predict the compliance with Social Engineering attacks, and detect human deception in human-robot interaction.

    \divider    
    
    \cvevent{Big Data Architect}{Data Reply}{Oct 2017 -- Nov 2018}{Milan \& Bologna, Italy}
    Design and development of architectures for the management and real-time processing of Big Data in the vehicle insurance field. Master's Degree dissertation project using DevOps principles and tools to fully automatise the development and deployment process of a movie recommendation service in a Big Data ecosystem.

    \divider    
    
    
\clearpage

\begin{fullwidth}

  Beyond the lab, I'm passionate about \textbf{cooking} and homebrewing beer—it’s a lot like programming! I also enjoy \textbf{playing board and card games} with friends every week. As a 	extit{Magic: The Gathering} enthusiast, I'm always on the lookout for interesting strategies to bring to the table.\\
  \divider\\
  
  \textbf{Foreign Languages: English (Professional Level)}\\
  \divider\\  
  
  \textbf{Programming Languages, Tools, Methodologies}: Python, C++, Scala, Arduino; Numpy, Pandas, OpenCV, OpenFace, OpenPose, Tensorflow, Keras, PyTorch, SciPy, Scikit-learn, Seaborn, Hugging Face, Spark; ROS1, ROS2, YARP, PyQt5/6, Jamovi, ChatGPT, Cloudera CDH, Kafka, Kudu, Ansible, Terraform.    

  \cvsection[page2sidebar]{Organized Events}
  \printbibliography[heading=pubtype,title=\empty, type=misc]

  \cvsection[page2sidebar]{Publications}
  \nocite{*}

  \printbibliography[heading=pubtype,title={\printinfo{\faFileTextO}{Journal}}, type=article]
  \divider

  \printbibliography[heading=pubtype,title={\printinfo{\faFileTextO}{Conference Proceedings}}, type=inproceedings]
  

\end{fullwidth}
\end{document}
